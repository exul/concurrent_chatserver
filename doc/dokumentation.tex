%        File: dokumentation.tex
%     Created: Thu Jun 05 14:05 PM 2012 C
%
\documentclass[a4paper]{article}

% german dictionary
\usepackage[ngerman]{babel}

% enconding
\usepackage[utf8]{inputenc}

% graphics
\usepackage{graphicx}

% handle positions
\usepackage{float}

% set borders
\usepackage{a4wide}

% use URLs
\usepackage{url}

% expand table formatting
\usepackage{array}

\floatstyle{ruled}
\newfloat{code}{thp}{lop}
\floatname{code}{Code}

% header and footer
\usepackage{fancyhdr}
\pagestyle{fancy}
\fancyhf{}

\fancyhead[L]{Concurrent Programming}
\fancyhead[R]{Zürcher Hochschule für Angewandte Wissenschaften}

\fancyfoot[L]{Andreas Brönnimann}
\fancyfoot[C]{\thepage}
\fancyfoot[R]{\today}
\renewcommand{\footrulewidth}{0.5pt}

% cover
\title {Seminararbeit Concurrent Programming\\
Multi-threaded Chatserver in C\\}
\author {Andreas Brönnimann\\
Zürcher Hochschule für Angewandte Wissenschaften\\
Dozent: Tomas Pospisek}
\date {\today}

\begin{document}

% show cover
\maketitle
\setcounter{page}{0}
\thispagestyle{empty}

\newpage

% define tableofcontents depth
\setcounter{tocdepth}{3}
\tableofcontents

\newpage

\section{Einführung}
\subsection{Projektbeschreibung}
Die Seminararbeit beinhaltet die Analye, Konzeption und Entwicklung eines multi-threaded Chat-Servers in C. Grundlegende Kenntnisse die dabei zur Anwendung kommen sind die Programmierung mit pthreads sowie die Benutzung des Socket-APIs.

\subsection{Ausgangslage}
In der Vorlesung haben die Studenten die Grundlagen der parallele Programmierung und der damit verbundenen Probleme kennengelernt. In Form einer Seminararbeit soll das erlernte Wissen nun in die Praxis umgesetzt werden.

\subsection{Ziele der Arbeit}
Entwicklung eines Chatservers in C, der in der Lage ist, mehrere Chat-Clients zu bedienen. Die Applikation muss serverseitig mit phtreads implementiert werden. Als Client kommt telnet bzw. netcat zum Einsatz. Das Ziel der Arbeit ist ein funktionaler Chatserver, der Nachrichten erfolgreich über das Netzwerk an andere, ebenfalls verbundene Chatclients, weitergeben kann.

\section{Bedienung}
\section{Probleme}
\subsection{TIME\_WAIT}
\subsection{Message während dem schreiben}
\section{Fazit}
\end{document}
