\documentclass[12pt]{beamer}
\usepackage[utf8]{inputenc}
\usepackage{lmodern}
\usepackage{german}
\usetheme{Berkeley}
\title[Seminararbeit]{Multi-threaded Chatserver in C}
\author{Andreas Brönnimann}
\institute{ZHAW - Zürcher Hochschule für Angewandte Wissenschaften}
\date{18.06.2012}

\setbeamerfont{footnote}{size=\tiny}
\setbeamertemplate{footline}[frame number]
\setbeamertemplate{navigation symbols}{}

\begin{document}

    \begin{frame}
        \titlepage
    \end{frame}

    \begin{frame}
        \frametitle{Ablauf}
        \tableofcontents
    \end{frame}

    \section{Implementation}
    \begin{frame}
        \frametitle{Implementation}
	    \begin{itemize}
		\item Netzwerk-Kommunikation: \texttt{getaddrinfo}
		    \begin{itemize}
			\item Unterstützt IPv4 und IPv6 
		    \end{itemize}
		\item LinkedList mit Mutex pro Element
		\item Kein Message Passing, da zu komplex
		\item String Handling: \texttt{strncpy}/\texttt{strncat}
            \end{itemize}
    \end{frame}

    \begin{frame}
    \frametitle{Implementation}
	\begin{itemize}
	    \item genutzte Compiler Optionen
		\begin{description} 
		    \item[-Wall:] um alle Warnungen auszugeben
		    \item[-Werror:] Warnungen werden zu Fehlern
		    \item[-g:] um das Programm debuggen zu können
		    \item[-lpthread:] Damit Pthreads verwendet werden können
	\end{description}

	\end{itemize}
    \end{frame}


    \section{Probleme}
    \begin{frame}[fragile]
	\frametitle{Problem}
	\begin{itemize}
	    \item TIME\_WAIT: Socket wiederverwenden
  	\end{itemize}

\begin{verbatim}
int on = 1;
int status = setsockopt(sfd, SOL_SOCKET, 
SO_REUSEADDR, (const char *) &on, sizeof(on));

if (status == -1) 
{   
    perror("setsockopt(...,SO_REUSEADDR,...)");
}
\end{verbatim}
\end{frame}
   
    \begin{frame}
	\frametitle{Probleme} 
	\begin{itemize}
	    \item Nachrichten während dem Schreiben
	\end{itemize}

	\begin{figure}[H]
	    \centering
	    \includegraphics[width=5cm]{gfx/chat_interrupt}
	    \label{img:chat_interrupt}
	\end{figure}
    \end{frame}

    \section{Bedienung}
    \begin{frame}
        \frametitle{Bedienung}
	Chatserver starten mit: \texttt{./chat localhost 54321}
	\bigskip
	Client verbinden: \texttt{telnet localhost 54321}
	\begin{description}
	    \item[/quit:] Chat verlassen
	    \item[/nick:] Nickname ändern \newline (bspw. \texttt{/nick John})
	    \item[/me:] Aktion beschreiben \newline (bspw. \texttt{/me geht C coden})
	\end{description}
    \end{frame}

    \section{Fazit}
    \begin{frame}
    \frametitle{Fazit}
	\begin{itemize}
	    \item Parallelität bringt mehr Komplexität als gedacht
	    \item in C muss man sich um alles kümmern (bspw. Speichermanagement)
	    \item Freud und Leid liegen nah beieinander
	\end{itemize}
    \end{frame}

    \section{Fragen}
    \begin{frame}
    \frametitle{Fragen}
        \begin{figure}[H]
	    \centering
	        \includegraphics[width=6cm]{gfx/questionmark}
        \end{figure}
    \end{frame}
\end{document}
